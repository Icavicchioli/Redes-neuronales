% !TEX TS-program = pdflatex
% !TEX encoding = UTF-8 Unicode

% This is a simple template for a LaTeX document using the "article" class.
% See "book", "report", "letter" for other types of document.

\documentclass[11pt]{article} % use larger type; default would be 10pt

\usepackage[utf8]{inputenc} % set input encoding (not needed with XeLaTeX)

%%% Examples of Article customizations
% These packages are optional, depending whether you want the features they provide.
% See the LaTeX Companion or other references for full information.

%%% PAGE DIMENSIONS
\usepackage{geometry} % to change the page dimensions
\geometry{a4paper} % or letterpaper (US) or a5paper or....
% \geometry{margin=2in} % for example, change the margins to 2 inches all round
% \geometry{landscape} % set up the page for landscape
%   read geometry.pdf for detailed page layout information

\usepackage{graphicx} % support the \includegraphics command and options

% \usepackage[parfill]{parskip} % Activate to begin paragraphs with an empty line rather than an indent

%%% PACKAGES
\usepackage{booktabs} % for much better looking tables
\usepackage{array} % for better arrays (eg matrices) in maths
\usepackage{paralist} % very flexible & customisable lists (eg. enumerate/itemize, etc.)
\usepackage{verbatim} % adds environment for commenting out blocks of text & for better verbatim
\usepackage{subfig} % make it possible to include more than one captioned figure/table in a single float
% These packages are all incorporated in the memoir class to one degree or another...

%%% HEADERS & FOOTERS
\usepackage{fancyhdr} % This should be set AFTER setting up the page geometry
\pagestyle{fancy} % options: empty , plain , fancy
\renewcommand{\headrulewidth}{0pt} % customise the layout...
\lhead{}\chead{}\rhead{}
\lfoot{}\cfoot{\thepage}\rfoot{}

%%% SECTION TITLE APPEARANCE
\usepackage{sectsty}
\allsectionsfont{\sffamily\mdseries\upshape} % (See the fntguide.pdf for font help)
% (This matches ConTeXt defaults)

%%% ToC (table of contents) APPEARANCE
\usepackage[nottoc,notlof,notlot]{tocbibind} % Put the bibliography in the ToC
\usepackage[titles,subfigure]{tocloft} % Alter the style of the Table of Contents
\renewcommand{\cftsecfont}{\rmfamily\mdseries\upshape}
\renewcommand{\cftsecpagefont}{\rmfamily\mdseries\upshape} % No bold!

%%% END Article customizations

%%% The "real" document content comes below...

\title{Trabajo práctico 1\\ Redes Neuronales y Aprendizaje Profundo}
\author{Ignacio Ezequiel Cavicchioli\\Padrón 109428\\icavicchioli@fi.uba.ar}
\date{} % Activate to display a given date or no date (if empty),
         % otherwise the current date is printed 

\begin{document}
\maketitle

\section{Ejercicio 1}

\subsection{Introducción}

\subsection{Resultados}

\subsection{Análisis}

\section{Ejercicio 2}

\subsection{Introducción}

\subsection{Resultados}

\subsection{Análisis}

\section{Conclusiones}

Leí tus notebooks y te dejo una devolución organizada:
Lo que está bien en tu análisis

* **Implementación clara**: programaste la regla de Hebb y la dinámica de actualización de Hopfield de manera correcta y modular.
* **Experimentos variados**: probaste con diferentes números de patrones y neuronas, lo que te permitió mostrar la relación entre capacidad y error.
* **Visualización**: los gráficos permiten ver cómo evoluciona la red y en qué punto empieza a fallar la memoria.
* **Discusión inicial**: mencionás la capacidad límite (aprox. `0.14N`) y observás cómo se degrada el rendimiento al aumentar la carga.

Aspectos en los que podrías ahondar

1. **Profundizar en la teoría**

   * Explicar mejor por qué la capacidad máxima se aproxima a `0.14N` (derivación a partir de resultados de Amit, Gutfreund y Sompolinsky).
   * Diferenciar entre *memorizar patrones* y *recuperarlos con ruido* (estabilidad de atractores vs. basins of attraction).

2. **Dinámica de actualización**

   * Comparar **actualización síncrona vs. asíncrona** y sus consecuencias en la convergencia.
   * Mostrar ejemplos donde la red entra en **ciclos** o estados espurios.

3. **Ruido y robustez**

   * Evaluar qué pasa si los patrones iniciales tienen cierto porcentaje de bits cambiados.
   * Graficar probabilidad de recuperación exitosa vs. nivel de ruido inicial.

4. **Estados espurios**

   * Mencionar y, si podés, mostrar ejemplos de **estados espurios mixtos** (combinaciones lineales de patrones almacenados).
   * Discutir qué implican para la capacidad real de la red.

5. **Extensiones posibles**

   * Comentar variantes como Hopfield continuo (con funciones sigmoides), o usar matrices de pesos con aprendizaje estocástico.
   * Mencionar relación con máquinas de Boltzmann y redes modernas de memoria asociativa.

---

Para tu **documento en LaTeX** te convendría estructurarlo así:

1. Introducción breve (qué es una red de Hopfield y para qué sirve).
2. Regla de aprendizaje (con ecuación).
3. Dinámica y convergencia.
4. Experimentos y resultados (capacidad, ruido, errores).
5. Limitaciones y próximos pasos (espurios, generalización).

¿Querés que te arme un **esqueleto en LaTeX** con estas secciones, listo para que pegues tus resultados?



\end{document}
