% !TEX TS-program = pdflatex
% !TEX encoding = UTF-8 Unicode

% This is a simple template for a LaTeX document using the ``article'' class.
% See ``book'', ``report'', ``letter'' for other types of document.

\documentclass[11pt]{article} % use larger type; default would be 10pt

\usepackage[utf8]{inputenc} % set input encoding (not needed with XeLaTeX)

%%% Examples of Article customizations
% These packages are optional, depending whether you want the features they provide.
% See the LaTeX Companion or other references for full information.

%%% PAGE DIMENSIONS
\usepackage{geometry} % to change the page dimensions
\geometry{a4paper} % or letterpaper (US) or a5paper or....
\geometry{margin=1in} % for example, change the margins to 2 inches all round
% \geometry{landscape} % set up the page for landscape
%   read geometry.pdf for detailed page layout information

\usepackage{graphicx} % support the \includegraphics command and options

% \usepackage[parfill]{parskip} % Activate to begin paragraphs with an empty line rather than an indent

%%% PACKAGES
\usepackage{booktabs} % for much better looking tables
\usepackage{array} % for better arrays (eg matrices) in maths
\usepackage{paralist} % very flexible & customisable lists (eg. enumerate/itemize, etc.)
\usepackage{verbatim} % adds environment for commenting out blocks of text & for better verbatim
\usepackage{subfig} % make it possible to include more than one captioned figure/table in a single float
% These packages are all incorporated in the memoir class to one degree or another...
\usepackage{enumitem}
\usepackage{amsmath}

% para los cuadritos en links
%\usepackage[linkbordercolor={0 0 1}, citebordercolor={0 1 0}, urlbordercolor={1 0 0}]{hyperref}
\usepackage[colorlinks=true, linkcolor=black, citecolor=green, urlcolor=red]{hyperref} % solo resalta
\usepackage[spanish]{babel}



%%% HEADERS & FOOTERS
\usepackage{fancyhdr} % This should be set AFTER setting up the page geometry
\pagestyle{fancy} % options: empty , plain , fancy
\renewcommand{\headrulewidth}{0pt} % customise the layout...
\lhead{}\chead{}\rhead{}
\lfoot{}\cfoot{\thepage}\rfoot{}

%%% SECTION TITLE APPEARANCE
\usepackage{sectsty}
\allsectionsfont{\sffamily\mdseries\upshape} % (See the fntguide.pdf for font help)
% (This matches ConTeXt defaults)

%%% ToC (table of contents) APPEARANCE
\usepackage[nottoc,notlof,notlot]{tocbibind} % Put the bibliography in the ToC
\usepackage[titles,subfigure]{tocloft} % Alter the style of the Table of Contents
\renewcommand{\cftsecfont}{\rmfamily\mdseries\upshape}
\renewcommand{\cftsecpagefont}{\rmfamily\mdseries\upshape} % No bold!

%%% END Article customizations

%%% The ``real'' document content comes below...



\title{Propuesta para Monografía\\ Redes Neuronales y Aprendizaje Profundo}
\author{Ignacio Ezequiel Cavicchioli\\Padrón 109428\\icavicchioli@fi.uba.ar}
% \date{} % Activate to display a given date or no date (if empty),
         % otherwise the current date is printed

\begin{document}
\maketitle

%\tableofcontents



\section{Introducción}

El presente documento detalla la propuesta ideada para la monografía requerida como parte del currículo de la materia \textit{Redes Neuronales y Aprendizaje Profundo}.

\section{Descripción de propuesta}

La propuesta busca explorar el uso de mapas auto-organizados (\textit{SOM}) en el contexto del control de sistemas por medio de las siguientes experiencias:

\begin{enumerate}
	\item \textbf{Uso del \textit{SOM} como explorador del espacio de estados de un sistema}

	La idea es entrenar un \textit{SOM} sobre el vector de estados de una planta a determinar con la intención de visualizar cómo un sistema puede ir de un punto de su espacio de estados a otro. Esto podría al usuario cierta intuición de cómo puede evolucionar el sistema, y conecta con \textbf{\textit{Gain Scheduling}}, explicado en los ``temas relacionados''.

	\item \textbf{Uso del \textit{SOM} como clonador/emulador de un controlador}

	Se entrena un \textit{SOM} sobre un vector de entrada conformado por el vector de estados de la planta y la referencia del lazo de control. Además, se guarda la salida del controlador como si fuera la \textit{label} de la data usada. Con esto último y la red entrenada, se asocia a cada neurona una respuesta del controlador típica según donde quedó ubicada.

	El último paso es reemplazar el controlador con el \textit{SOM}. Con una referencia y vector de estados actuales, el \textit{SOM} va a producir una acción de control dictada por la \textbf{BMU}, que se espera aproxime la respuesta original. Se probará con varias velocidades de actualización del control, porque va a ser discreto, no continuo, y se espera que haya fuerte dependencia con la densidad de neuronas.

	Esta segunda propuesta busca ver si el \textit{SOM} es capaz de clonar un controlador, que, habiendo investigado un poco, emularía una estrategia de control real, citada en los ``temas relacionados''.

\end{enumerate}

En principio se usaría una planta SISO (1 entrada, 1 salida) con estrategias de control del estilo PID o armadas a mano por \textit{pole placement}.

\newpage

\section{Temas relacionados}

Los temas relacionados con esta monografía son:
\begin{itemize}
	\item \textbf{\textit{Gain Scheduling}}:

	Es un \textit{approach} para controlar sistemas no lineales en el que se cambia el controlador en base a las condiciones operativas actuales. Sirve para cuando se quiere usar controladores lineales (linealizados en un punto del espacio de estados) en sistemas que tienen muchas condiciones de operación, como los aviones a diferentes alturas. Esto es parte de lo denominado ``control adaptativo''.

	Estudiar el espacio de estados de un sistema en todo su rango de operación podría ayudar a encontrar esas regiones para las cuales generar controles. También se podría usar un \textit{SOM} para saber en qué región del espacio se está (según la \textbf{BMU}), y hacer el cambio de controlador.

	\item \textbf{\textit{Imitation learning}}:

	La monografía se relaciona con este tema ya que se está entrenando un \textit{SOM} en base a demostraciones de un controlador. Aunque el resultado final seguramente no sea un clon perfecto del controlador, creo que va a emularlo razonablemente bien.
\end{itemize}

\end{document}
