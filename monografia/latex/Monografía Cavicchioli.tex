% !TEX TS-program = pdflatex
% !TEX encoding = UTF-8 Unicode

% This is a simple template for a LaTeX document using the ``article'' class.
% See ``book'', ``report'', ``letter'' for other types of document.

\documentclass[11pt]{article} % use larger type; default would be 10pt

\usepackage[utf8]{inputenc} % set input encoding (not needed with XeLaTeX)

%%% Examples of Article customizations
% These packages are optional, depending whether you want the features they provide.
% See the LaTeX Companion or other references for full information.

%%% PAGE DIMENSIONS
\usepackage{geometry} % to change the page dimensions
\geometry{a4paper} % or letterpaper (US) or a5paper or....
\geometry{margin=1in} % for example, change the margins to 2 inches all round
% \geometry{landscape} % set up the page for landscape
%   read geometry.pdf for detailed page layout information

\usepackage{graphicx} % support the \includegraphics command and options

% \usepackage[parfill]{parskip} % Activate to begin paragraphs with an empty line rather than an indent

%%% PACKAGES
\usepackage{booktabs} % for much better looking tables
\usepackage{array} % for better arrays (eg matrices) in maths
\usepackage{paralist} % very flexible & customisable lists (eg. enumerate/itemize, etc.)
\usepackage{verbatim} % adds environment for commenting out blocks of text & for better verbatim
\usepackage{subfig} % make it possible to include more than one captioned figure/table in a single float
% These packages are all incorporated in the memoir class to one degree or another...
\usepackage{enumitem}
\usepackage{amsmath}
\usepackage{siunitx}
% para los cuadritos en links
%\usepackage[linkbordercolor={0 0 1}, citebordercolor={0 1 0}, urlbordercolor={1 0 0}]{hyperref}
\usepackage[colorlinks=true, linkcolor=black, citecolor=green, urlcolor=red]{hyperref} % solo resalta
\usepackage[spanish]{babel}



%%% HEADERS & FOOTERS
\usepackage{fancyhdr} % This should be set AFTER setting up the page geometry
\pagestyle{fancy} % options: empty , plain , fancy
\renewcommand{\headrulewidth}{0pt} % customise the layout...
\lhead{}\chead{}\rhead{}
\lfoot{}\cfoot{\thepage}\rfoot{}

%%% SECTION TITLE APPEARANCE
\usepackage{sectsty}
\allsectionsfont{\sffamily\mdseries\upshape} % (See the fntguide.pdf for font help)
% (This matches ConTeXt defaults)

%%% ToC (table of contents) APPEARANCE
\usepackage[nottoc,notlof,notlot]{tocbibind} % Put the bibliography in the ToC
\usepackage[titles,subfigure]{tocloft} % Alter the style of the Table of Contents
\renewcommand{\cftsecfont}{\rmfamily\mdseries\upshape}
\renewcommand{\cftsecpagefont}{\rmfamily\mdseries\upshape} % No bold!

%%% END Article customizations

%%% The ``real'' document content comes below...



\title{
	\vspace{-2cm} % Ajusta este valor para subir/bajar todo el bloque
	\centering
	\LARGE \textbf{Monografía Final} \\[0.8cm]
	\Huge \textbf{Mi título} \\[1.5cm]
	\large
	\textbf{Universidad de Buenos Aires} \\
	Facultad de Ingeniería \\[0.5cm]
	\normalsize
	\begin{tabular}{r l}
		\textbf{Alumno:} & Ignacio Ezequiel Cavicchioli \\
		\textbf{Padrón:} & 109428 \\
		\textbf{Email:} & icavicchioli@fi.uba.ar
	\end{tabular}
}
\author{} % Dejamos vacío porque el autor ya está en el título
\date{\vspace{0.5cm} \today} % Fecha actual

\begin{document}
\maketitle

%\tableofcontents

\begin{abstract}
	\noindent
Lorem ipsum dolor sit amet, consetetur sadipscing elitr, sed diam nonumy eirmod tempor invidunt ut labore et dolore magna aliquyam erat, sed diam voluptua. At vero eos et accusam et justo duo dolores et ea rebum. Stet clita kasd gubergren,

	\vspace{0.5em}
	\noindent
	\textbf{Palabras clave:} Lorem ipsum dolor sit amet.
\end{abstract}

\vspace{1cm}

% TABLA DE CONTENIDOS
\tableofcontents
\newpage

% CONTENIDO PRINCIPAL
\section{Introducción}
\label{sec:introduccion}

Las redes neuronales en sus múltiples formas constituyen sistemas no lineales con un amplio alcance de aplicación en campos como la biología, neurociencia y, al que se avoca este trabajo, aprendizaje automático (\textit{machine learning}, ML). En particular, nos interesa centrarnos en las aplicaciones de las redes neuronales en el campo de control automático. Esta doctrina se encarga del diseño sistemas para regular, guiar o estabilizar procesos de manera autónoma, mediante la realimentación y corrección continua de errores.

La denominada ``caja de herramientas'' de aquellos en el área de control está compuesta por ciertos artefactos matemáticos que permiten encarar estos problemas, como la linealización de un sistema, el control PID, realimentación de estados, \textit{loop-shaping}, observadores, etc. Lo que no se ha tocado en las materias de control son las estrategias no lineales. En líneas generales, todos los sistemas reales exhiben cierto grado de no linearidad (citar CONTROL SYSTEM DESIGN Goodwin  Graebe Salgado, p551, cap19), lo que implica que las estrategias de control lineales son válidas siempre que las no linealidades sean despreciables. Análogamente, las herramientas de identificación de sistemas basadas en la linealización de un sistema fallaran en modelar las no linealidades de estos (Como PCA vs los \textit{autoencoders}).

Este trabajo va a trabajar sobre el uso de redes neuronales en la doctrina de control automático, como identificacores de sistemas, controladores

\textbf{VER SI QUERËS METER ALGO DEL SOM QUE SALIÖ MAL Y DE OBERVADORES Y DE GAIN SCHEDULING}

\section{Sistema elegido}

Que
tamaños
Porque
lo bueno
lo malo
la matemática -> espacio de estados no lineal y linealizado, transferencia (lineal).
el simulink


El sistema elegido está compuesto por 2 tanques de agua de dimensiones diferentes. Tanto la tubería que une los tanques como la que sale tienen cierto coeficiente hidráulico asociado a su geometría. Los valores se eligieron para que el sistema tenga sentido físico aunque no es un requisito. Se podría pensar que el sistema actúa como una cisterna amortiguadora de fluctuaciones en el caudal seguido de un reservorio que ajusta el caudal de salida.

\begin{figure}[h!]
	\centering
	\includegraphics[width=0.7\linewidth]{imgs/tanques}
	\caption{Sistema elegido}
	\label{fig:tanques}
\end{figure}

La figura \ref{fig:tanques} muestra el sistema recién descripto, agregando los caudales de entrada y salida $u$ e $y$. Ambos caudales son muestreados a \SI{1}{\Hz}, que debería ser más que suficiente para este tipo de dinámicas lentas.

Las razones por la cuales se eligió este sistema son:
\begin{itemize}
	\item Simplicidad: Es un sistema de 2 estados, simple de modelar, linealizar, simular y controlar. Las redes quese prueben deberían poder con este problema.
	\item No linealidad: Como se va a ver en el inciso matemático, el sistema no es lineal, que sería un requisito si se está intentando evaluar la capacidad de copiar no linealidades de las redes neuronales.
	\item Realismo: Se prefirió elegir un sistema que sea fácil de entender pero real, no una abstracción de un sistema más complejo.
\end{itemize}


\subsection{Modelo matemático}

El modelado de este sistema hace uso de varias leyes físicas de la hidráulica. Primero se plantea que el líquido es incompresible, por lo que el volumen solo varía si los caudales de entrada y salida no son iguales.

\begin{equation}
	\frac{dV}{dt} = \sum q_{in} - \sum q_{out}
	\label{eq:1}
\end{equation}


Además, el volumen es función del área del tanque y su superficie (en este caso que el área es independiente del nivel de agua). Podemos derivar respecto del tiempo.

\begin{equation}
	V(t) = A \cdot h(t) \overset{d/dt}{\longrightarrow } \frac{dV}{dt} = A \cdot \frac{dh(t)}{dt}
	\label{eq:2}
\end{equation}

Luego se aplica la ley de caudal, que relaciona el caudal entre 2 puntos con la diferencia de nivel entre ellos mismos.

\begin{equation}
	q = k\,\sqrt{2g}\,\sqrt{\Delta h}
	\label{eq:3}
\end{equation}

El caudal entre los tanques resulta:

\begin{equation}
	q_{12}(t) = k_{12}\,\sqrt{2g}\,\sqrt{h_1(t) -h_2(t)}
	\label{eq:4}
\end{equation}

El caudal de salida del segundo tanque, que también es la salida $y$, es:
\begin{equation}
	q_{2}(t) = y(t) = k_{2}\,\sqrt{2g}\,\sqrt{h_2(t)}
		\label{eq:5}
\end{equation}


Ahora, se plantea un balance en cada tanque igualando \eqref{eq:1} y \eqref{eq:2}.

\begin{equation}
	A_1 \cdot \frac{dh_1}{dt} = u(t) - q_{12}(t)
	\label{eq:6}
\end{equation}

\begin{equation}
	A_2 \cdot \frac{dh_2}{dt} = q_{12}(t) - q_{2}(t) =  q_{12}(t) - y(t)
	\label{eq:7}
\end{equation}

sustituyendo con \eqref{eq:4} :

\begin{equation}
	A_1 \cdot \frac{dh_1}{dt} = u(t) - k_{12}\,\sqrt{2g}\,\sqrt{h_1(t) -h_2(t)}
	\label{eq:8}
\end{equation}

\begin{equation}
	A_2 \cdot \frac{dh_2}{dt} = k_{12}\,\sqrt{2g}\,\sqrt{h_1(t) -h_2(t)} -  k_{2}\,\sqrt{2g}\,\sqrt{h_2(t)}
	\label{eq:9}
\end{equation}

Con todo esto se plantea el espacio de estados\textbf{ no lineal} y continuo con la forma de a continuación:

\begin{equation}
	\frac{d}{dt}
	\begin{bmatrix}
		h_1
		\\
		h_2
	\end{bmatrix} =
	\begin{bmatrix}
		(u - k_{12}\,\sqrt{2g}\,\sqrt{h_1 -h_2})\cdot \frac{1}{A_1}
		\\
		(k_{12}\,\sqrt{2g}\,\sqrt{h_1 -h_2} -  k_{2}\,\sqrt{2g}\,\sqrt{h_2})\cdot \frac{1}{A_2}
	\end{bmatrix}
	\label{eq:ss1}
\end{equation}


\begin{equation}
	y=k_{2}\,\sqrt{2g}\,\sqrt{h_2}
	\label{eq:ss2}
\end{equation}

Las expresiones \eqref{eq:ss1} y \eqref{eq:ss2} se van a linealizar en torno al punto de equilibrio $(x_e,u_e)$ y las constantes físicas indicadas abajo.

\begin{itemize}
	\item $g = 9.81$
	\item $A_1 = 0.015$, $A_2 = 0.02$
	\item $k_{12} = 5 \times 10^{-4}$, $k_2 = 1 \times 10^{-3}$
	\item $h_{1s} = 1.019$, $h_{2s} = 0.204$
	\item $u_s = 0.002$
\end{itemize}

%g   = 9.81;
%A1  = 0.015;
%A2  = 0.02;
%k12 = 5e-4;
%k2  = 1e-3;
%h1s = 1.019;
%h2s = 0.204;
%us  = 0.002;

Las variables de estado elegidas se redefinen como variaciones en torno a ese mismo estado, por lo que de ahora en más las alturas $h_1$ y $h_2$ no son las mismas que en la planta no lineal. El proceso arranca planteando la linealización en sí, que se ve en la ecuación \eqref{eq:12}. La expresión \eqref{eq:13} se cumple por definición del punto $(x_e,u_e)$. En \eqref{eq:14a} y \eqref{eq:14b} se obtiene la matriz A del espacio de estados lineal. En \eqref{eq:15} se obtiene la matriz B, y en \eqref{eq:17}, la C.

Las expresiones \eqref{eq:16} y \eqref{eq:17} constituyen el espacio de estados lineal para la dinámica de los tanques.

\begin{equation}
\overset{\,\circ }{X} = \overset{\,\circ }{\begin{bmatrix}
		h_1
		\\
		h_2
\end{bmatrix}} =f(x_e,u_e) + \frac{df}{d x}|_{(x_e,u_e)} (x-x_e,u-u_e) +  \frac{df}{d u}|_{(x_e,u_e)} (x-x_e,u-u_e)
\label{eq:12}
\end{equation}

\begin{equation}
f(x_e,u_e)=0
\label{eq:13}
\end{equation}

\begin{equation}
\frac{df}{d x}|_{(x_e,u_e)} (x-x_e,u-u_e) = \begin{bmatrix}
	\frac{-k_{12} \sqrt{2g}}{A_1 \cdot 2 \sqrt{h_1-h_2}}
	&
	\frac{k_{12} \sqrt{2g}}{A_1 \cdot 2 \sqrt{h_1-h_2}}
	\\
	\frac{k_{12} \sqrt{2g}}{A_2 \cdot 2 \sqrt{h_1-h_2}}
	&
	\frac{-k_{12} \sqrt{2g}}{A_2 \cdot 2 \sqrt{h_1-h_2}} - \frac{k_{2} \sqrt{2g}}{A_2 \cdot 2 \sqrt{h_2}}
\end{bmatrix}
|_{(x_e,u_e)}
\label{eq:14a}
\end{equation}

\begin{equation}
	\frac{df}{d x}|_{(x_e,u_e)} (x-x_e,u-u_e) =
	 \simeq
	\begin{bmatrix}
		-0.0818
		&
		0.0818
		\\
		0.0613
		&
		-0.1841
	\end{bmatrix}
	\label{eq:14b}
\end{equation}


\begin{equation}
\frac{df}{d u}|_{(x_e,u_e)} (x-x_e,u-u_e) = \begin{bmatrix}
	\frac{1}{A_1}
	\\
	0
\end{bmatrix}|_{(x_e,u_e)}=
\begin{bmatrix}
	66.67
	\\
	0
\end{bmatrix}
\label{eq:15}
\end{equation}


\begin{equation}
\overset{\,\circ }{\begin{bmatrix}
		h_1
		\\
		h_2
\end{bmatrix}} =
\begin{bmatrix}
	-0.0818
	&
	0.0818
	\\
	0.0613
	&
	-0.1841
\end{bmatrix}
\begin{bmatrix}
	h_1
	\\
	h_2
\end{bmatrix}
+
\begin{bmatrix}
	66.67
	\\
	0
\end{bmatrix} u
\label{eq:16}
\end{equation}

\begin{equation}
y = \begin{bmatrix}
	0 & 0.00246
\end{bmatrix} \begin{bmatrix}
	h_1
	\\
	h_2
\end{bmatrix}
\label{eq:17}
\end{equation}

\subsection{Simulaciones}

Para esta monografía se decidió hacer uso de \textit{scripts} \textit{MATLAB} y el entorno de \textit{Simulink} debido a la versatilidad que trae en lo que es simulación de sistemas, además que es la herramienta usada en las materias de control automático. La figura \ref{fig:pnolin} muestra el sistema no lineal armado. En pocas palabras, usa las constantes definidas en el \textit{workbench}, la entrada $u$ y los estados integrados para calcular el siguiente estado de la planta.

\begin{figure}[h!]
	\centering
	\includegraphics[width=0.6\linewidth]{imgs/P_no_lin}
	\caption{Planta no lineal armada en \textit{simulink}}
	\label{fig:pnolin}
\end{figure}


\begin{figure}[h!]
	\centering
	\includegraphics[width=0.8\linewidth]{imgs/P_no_lin_muestreada}
	\caption{Muestreo de planta no lineal}
	\label{fig:pnolinmuestreada}
\end{figure}

La figura \ref{fig:pnolinmuestreada} muestra el sistema de \textit{simulink} que emula el muestreo de la planta. De entrada $u$ se usó una secuencia de escalones de amplitudes aleatorias alrededor del punto operativo. Esto se hizo para poder tener una buena variedad de respuestas al escalón, que se espera ayude a que la red aprenda la dinámica.

Como el muestreo se realizó a \SI{1}{\Hz}, se decidió generar un \textit{dataset} de 4 horas (14400 muestras), y otro de 1 hora (3600 muestras). La intención es observar cuanto empeora la \textit{performance} de la red con menos muestras de entrenamiento. Se usó una partición de \textit{train/test/}








\section{Identificación de sistemas}
\label{subsec:ids}

La idea es que la red copie la no linealidad de la planta

1 armé esa red rancia que en fre run anda para el orto pero en predicción de a 0 anda bien

Es decir, si se alimenta con los estados de la planta no lineal es un excelente predictor, pero en free-run, realimentado con si mismo, diverge en X tiempo. Para remediar esto, se buscan alternativas.

La primera es espacio de estados no lineal. hay un mapeo entre estado actual al siguiente dado por una función no lineal, y un mapeo de estado actual a salida tambien dado por funci¿ón no lineal. La idea es que el sistema aprenda esas funciones en vez de todo el proceso de integración implicito en la planta no lineal

para resolver eso hacemos:

luegoe spacio de estados no lineal
parte linealizada y luego entrenas la no lineal para que te de la parte no lineal y cerras el lazo con lo no lineal + lineal a ver que tal



\section{Control de sistemas}
\label{sec:control}

\section{Observadores}
\label{subsec:obs}

\section{Misceláneos}
\label{subsec:misc}



\section{Conclusiones}
\label{sec:conclusiones}



\href{https://en.wikipedia.org/wiki/Neural_network_(machine_learning)}{text}
\href{https://en.wikipedia.org/wiki/Machine_learning}{text}
\href{https://en.wikipedia.org/wiki/Neural_network}{text}
\href{https://la.mathworks.com/help/deeplearning/ug/introduction-to-neural-network-control-systems.html}{text}
\href{https://www.sciencedirect.com/topics/engineering/neural-network-controller}{text}
\href{}{text}
\href{}{text}

CONTROL SYSTEM
DESIGN
Graham C. Goodwin1
Stefan F. Graebe2
Mario E. Salgado3

\end{document}
